\documentclass[report,a4paper,12pt]{jlreq}

% ===== 日本語フォント & 基本パッケージ =====
\usepackage{luatexja}
\usepackage{luatexja-fontspec}     % 日本語フォント指定
\usepackage[margin=25mm]{geometry} % 単純・確実に余白指定
\usepackage{graphicx}              % 図
\usepackage{amsmath,amssymb}       % 数式
\usepackage{booktabs}              % 表の罫線
\usepackage{siunitx}               % 単位
\usepackage{hyperref}              % しおり・リンク
\usepackage{url}
\usepackage[numbers,sort&compress]{natbib} % 文献(数値式)

% ===== フォント(macOSの例:ヒラギノ)=====
\setmainjfont{Hiragino Mincho ProN}
\setsansjfont{Hiragino Kaku Gothic ProN}

% 欧文(お好みで)
\setmainfont{Times New Roman}
\setsansfont{Arial}
\setmonofont{Menlo}

% ===== 体裁の微調整 =====
\numberwithin{figure}{chapter}
\numberwithin{table}{chapter}
\numberwithin{equation}{chapter}
\hypersetup{
  colorlinks=true,
  linkcolor=black,
  citecolor=black,
  urlcolor=black,
  pdftitle={卒業論文タイトル},
  pdfauthor={山中 春樹}
}

% ===== 表紙情報 =====
\title{卒業論文タイトル(日本語)}
\author{山中 春樹}
\date{2025年2月}

\begin{document}

% ===== 表紙 =====
\begin{titlepage}
  \centering
  {\LARGE 卒業論文 \par}
  \vspace{20mm}
  {\Huge \bfseries 卒業論文タイトル(日本語)\par}
  \vspace{20mm}
  {\Large 山中 春樹 \par}
  \vspace{10mm}
  {\large 指導教員:○○ 教授\par}
  \vfill
  {\large 〇〇大学 〇〇学部 〇〇学科 \par}
  {\large 2025年2月 \par}
\end{titlepage}

% ===== 和文要旨 =====
\chapter*{要旨}
本研究では〜〜〜(200〜400字程度)。研究目的、手法、結果、結論を簡潔にまとめる。
\vspace{2\baselineskip}

% ===== 英文要旨 =====
\chapter*{Abstract}
This thesis investigates ... (about 150–250 words).
変えてみたぜ。

% 目次・図目次・表目次
\tableofcontents
\listoffigures
\listoftables

% ===== 本文(章ごとに分割)=====
\chapter{序論}
\label{chap:intro}

本章では研究の背景・目的・貢献・論文構成について述べる。

\section{背景}
……本文……

\section{目的}
……本文……

\section{本論文の構成}
本論文の構成は以下の通りである。第\ref{chap:related}章で関連研究,%
第\ref{chap:method}章で提案手法,%
第\ref{chap:exp}章で実験,%
第\ref{chap:discuss}章で考察,%
第\ref{chap:conclusion}章で結論を述べる。

\chapter{関連研究}\label{chap:related}
関連研究を体系的に整理・比較する。

\section{骨格認識}
この研究は \cite{doe2023awesome} と \cite{lamport1994latex} に基づいて行われる。
\chapter{提案手法}\label{chap:method}
手法の概観,詳細,アルゴリズム,計算量,実装条件など。

\chapter{実験}\label{chap:exp}
\section{設定}
データセット,評価指標,ハイパーパラメータ等。

\section{結果}
図表を交えて結果を報告する(例:図\ref{fig:example})。

\begin{figure}[t]
  \centering
  \includegraphics[width=.7\linewidth]{example-image-a}
  \caption{サンプル図(差し替えてください)}
  \label{fig:example}
\end{figure}

\chapter{考察}\label{chap:discuss}
結果の解釈,限界,失敗事例,将来課題など。

\chapter{結論}\label{chap:conclusion}
本研究のまとめと今後の展望。


% ===== 謝辞 =====
\chapter*{謝辞}
本研究の遂行にあたり,ご指導いただいた〜〜〜教授に深く感謝いたします。研究室の皆様,家族にも感謝します。

% ===== 参考文献 =====
\bibliographystyle{plainnat}   % 要件に合わせて変更可(unsrtnat等)
\bibliography{refs}

% ===== 付録(必要なら)=====
% \appendix
% \chapter{追加実験}
% 付録本文…

\end{document}
